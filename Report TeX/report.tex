\documentclass[UTF8,12pt]{article}
\usepackage{indentfirst}

\title{COMP30024 Project Part A \\
       Report}
\author{Zheping Liu, 683781 \\
        Bohan Yang, 814642}
\date{}

\begin{document}
    \maketitle
    \section{Formulation of the game as a search problem}
    The states of the game is structured as a dictionary with 
    elements {current positions of all pieces, previous action 
    to get to this state, current positions of all blocks, 
    previous state, colour of the pieces}.
    The initial state is the constructed using the given file, with no previous
    action and previous state.
    The actions include JUMP, MOVE, and EXIT. 
    The goal test is achieved by testing if all pieces are removed 
    from the game board. 
    As the final goal requires an extra step from the EXIT position, 
    the sub goal is for any piece to get to any of the EXIT positions. 
    The path cost for each action is simply assumed as 1 in this program.

    \section{Search Algorithm}
    The search algorithm used by the program is A*, with the modified 
    Manhattan distance for hexagonal grids. This algorithm is chosen for 
    the optimality and relatively short time consumption. 
    The time complexity for A* search is exponential in (relative error in 
    the heuristic x length of solution). 
    This algorithm is also complete, unless there are infinitely many nodes with f <= f(G), 
    which does not appear in this program. 
    The space required for A* search is to keep all expanded nodes (closed list)
    and all nodes to be expanded (open list) in memory.
    The modified Manhattan distance heuristic is admissible as it is the sum of 
    the distance from all pieces to their closest EXIT positions respectively. 
    The states to be visited are stored in a priority queue, which takes the 
    shortest heuristic distance as their priority, the nodes with shorter heuristics
    will be opened (expanded) first.

    \section{Time and Space Complexity}
    The space complexity for this program is very high due to the structure of 
    “states” in our program. “States” is a dictionary with one of the elements 
    being previous state. Previous state is a nested attribute that stores all previous states before 
    the current state, therefore, accumulates magnificently as more actions are taken. 
    The time complexity of this program will increase a lot as the number of pieces
    increase in the game board. This is due to there are many possible successors
    can be generated at each state. Also, since our heuristic considers the total
    distance by summing up all shortest distances from every piece to sub-goal,
    many states will have same heuristics which makes the program will not always
    expand the optimal node.


\end{document}